\documentclass[
a4paper, 							% Papierformat
%10pt,								% Schriftgr��e (12pt, 11pt (Standard))
%twoside,							% Doppelseiten
titlepage,						% Titelei auf eigener Seite
normalheadings,				% �berschriften etwas kleiner (smallheadings)
%idxtotoc,						% Index im Inhaltsverzeichnis
%liststotoc,					% Abb.- und Tab.verzeichnis im Inhalt
%bibtotoc,						% Literaturverzeichnis im Inhalt
%leqno,   						% Nummerierung von Gleichungen links
%fleqn,								% Ausgabe von Gleichungen linksb�ndig
%draft								% �berlangen Zeilen in Ausgabe gekennzeichnet
]
{scrreprt}
\usepackage[ngerman]{babel}
\usepackage[isolatin]{inputenc}
\usepackage{graphicx}
\usepackage{url}

\begin{document}

\tableofcontents

\chapter{Einf�hrung}
\label{chap:einfuehrung}
Modelle als Abbildung der Realit�t\\
Schwierigkeiten beim finden von Testf�llen f�r komplexe Anwendungen

\begin{figure}[htbp]%
\centering
\includegraphics[width=150px]{rollikids_logo.png}%
\caption{dies ist ein testbild}%
\label{testbild}%
\end{figure}

Mehr Text gibts in dem erw�hnten Kapitel \ref{chap:einfuehrung} auf Seite \pageref{chap:einfuehrung}.

\chapter{Abstract State Machines}
Yuri Gurevich, und wie Algorithmen einfach mit Zust�nde repr�sentiert werden k�nnen

\chapter{Entwicklung von Spec Explorer}

\section{AsmL}

\section{AsmL-T}

\section{Spec$\#$}

\section{PEX}	

\chapter{Modellbasiertes Testen}

\section{Erkl�rung}

\section{Schwierigkeiten}

\subsection{State Explosion}

\subsection{Spezifikation}

\subsection{Zust�nde und Szenarien}

\subsection{IDE Unterst�tzung}

\subsection{Testfallerzeugung}

\chapter{Spec Explorer 2010}

\section{Die L�sung}

\section{Umgebung}

\subsection{Implementierung}

\subsection{Modell}

\subsection{Konfiguration}

\section{Exploration}

\section{Szenarien}

\section{Testf�lle}

\chapter{Visual Studio Integration}

\section{Exploration Manager}

\section{UML Extensions}

\section{Testing Tools}

\chapter{Einsatz bei Microsoft}

\chapter{Fazit}
Dies ist ein ganz kurzer Beispieltext \cite{Boerger}.

\bibliography{SpecExplorer}
\bibliographystyle{alpha}

\end{document}
